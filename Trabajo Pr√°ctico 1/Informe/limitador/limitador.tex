\chapter{Circuito Limitador Básico}
El circuito limitador básico está compuesto por una resistencia en serie y dos Diodos Zener enfrentados, configurados como se observa en la figura \ref{fig:limitador_basico}.

\begin{figure}[ht]
    \begin{center}
        \begin{circuitikz}[american voltages]
    \draw
    (0,0) node[ocirc](Vi-){}
    (0,4) node[ocirc](Vi+){}
    (5,0) node[ocirc](Vo-){}
    (5,4) node[ocirc](Vo+){}
    (Vi+) to[R=$R_s$,-*] ++(3,0)
    to[zzD*, l=$D_{z1}$] ++(0,-2)
    (Vi-) to[short, -*] ++(3,0)
    to[zzD*, l=$D_{z2}$] ++(0,+2)
    (Vo+) -- ++(-2,0)
    (Vo-) -- ++(-2,0)
    (Vi+) to[open, v=$V_i$] (Vi-)
    (Vo+) to[open, v=$V_o$] (Vo-)
    ;
\end{circuitikz}
\caption{Circuito Limitador Básico}
        \label{fig:limitador_basico}
    \end{center}
\end{figure}

\section{Funcionamiento}
Para analizar la operación del circuito se puede pensar en los siguientes casos:

\begin{enumerate}
    \item $|V_i| \leq V_f$
    \item $V_f < |V_i| \leq V_z + V_f$
    \item $V_z + V_f < |V_i|$ 
\end{enumerate}

En el caso 1, la tensión no es suficiente ni siquiera para polarizar el Diodo 1 en directa y no fluirá la corriente. Por lo tanto, la tensión de la salida seguirá a la de entrada.

En el caso 2, la tensión es suficiente para polarizar el Diodo 1 en directa, y el Diodo 2 en inversa. Sin embargo, esta polarización inversa no será suficiente para que el diodo entre en la zona de operación Zener, por lo cual la corriente que fluya será despreciable y la tensión de salida será aproximadamente igual a la de entrada.

En el caso 3, la tensión ya es suficiente no solo para polarizar el Diodo 1 en directa, y el Diodo 2 operará en modo Zener. Cuando esto ocurre, el Diodo zener fija su tensión en $V_z$ y por lo tanto $V_o = V_z + V_f$.

Como los diodos están enfrentados, la transferencia de la tensión será simétrica respecto el origen.

\section{Selección de Componentes}

Uno de los principales parámetros a considerar para el diseño es la máxima potencia que puede disipar el Diodo Zener. Con ese dato se calculó la corriente máxima que puede fluir a través del Diodo.

Para conseguir este límite de corriente, conociendo la tensión máxima que se utilizará (10V) se calculó la $R_s$ mínima para fijar este límite.

<<Fórmulas, Fórmulas, Fórmulas>>

<<Tabla de Componentes elegidos, Diodos, Resistencias, etc.>>

\section{Resultados}

\subsection{Teóricos}
<<Cálculo y diagrama de cómo quedaría teóricamente el circuito>>
\subsection{Simulación}

\subsection{Prácticos}
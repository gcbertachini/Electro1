\chapter{Autopolarización de Base}
\section{Análisis teórico}
La segunda sección de este trabajo práctico consistirá en el análisis de la autopolarización de 
base para un transistor bipolar de un juntura. Primeramente se estudiará la base del funcionamiento
en dicha configuración para luego aplicarlo a diversos modelos de transistores provistos 
por la cátedra. \par 
El circuito inicial propuesto es el siguiente:

\begin{figure}[H]
    \centering
    \begin{center}
        \includegraphics[width=0.3\textwidth]{polarizacion_de_base/Imagenes/circuito_polarizacion_base.png}
        \label{circuito_ejercicio_2}
    \end{center}
    \vspace{-8mm}
    \caption{Circuito para Autopolarización de base}
\end{figure}
Considerando que tanto la base como el colector comparten un node común, el mismo se puede 
separar como si fueran dos fuentes diferentes, una vez realizado este proceso se aplica 
el teorema de Thevenin obteniéndose el equivalente:
\begin{equation}
    V_{th}= V_{cc} \hspace{20mm} R_{th}=R_B 
    \label{eq_thevenin} 
\end{equation}
Aplicando las ecuaciones despejadas de [\ref{eq_thevenin}] en el circuito propuesto [\ref{circuito_ejercicio_2}]
se simplifica el circuito a la expresión:

\begin{figure}[H]
    \centering
    \begin{center}
        \includegraphics[width=0.3\textwidth]{polarizacion_de_base/Imagenes/circuito_polarizacion_base_thevenin.png}
        \label{circuito_ejercicio_2_thevenin}
    \end{center}
    \caption{Circuito para Autopolarización de base con Thevenin}
\end{figure}
Por otro lado, para este caso en particular se puede apreciar la presencia de la 
resistencia en el emisor ()$R_E$), lo que nos permite protegernos de las grandes fluctuaciones 
que puede tener la ganancia de corriente en un BJT ($h_{fe}$). De esta manera, se pueden 
despejar los parámetros de la malla de salida en función de los parámetros de la malla 
de entrada. Obteniéndose las siguientes expresiones: \par 
\begin{equation}
    \vspace{8mm}
    I_C=\frac{V_{th}-V_{BE}}{R_E\left(\frac{1+h_{fe}}{h_{fe}}\right)+\frac{R_{th}}{h_{fe}}}
    \label{eq_I_c}
\end{equation}

\begin{equation}
    V_{CE}=V_{CC}-I_C(R_C+R_E)
    \label{eq_V_CE}
\vspace{8mm}
\end{equation}


\section{Casos de aplicación}

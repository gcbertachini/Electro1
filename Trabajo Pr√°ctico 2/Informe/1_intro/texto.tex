\chapter{Introducción}

El objetivo de este trabajo fue comprobar en la práctica algunos de los aspectos más destacados de un circuito estudiado durante la cátedra de Electrónica I, dentro del marco de una simulación con recursos escasos.

En este caso, se estudió el comportamiento de un Par Darlington con una carga activa, en la forma de un espejo de corriente implementado con un par de transistores MOSFET.

\begin{figure} [ht]
    \centering
    \begin{minipage}[b]{0.48\textwidth}
        \centering
        \begin{circuitikz}[american voltages]
            \draw
            (0,1) node[npn](Q1){$Q_1$}
            (1,0) node[npn](Q2){$Q_2$}
            (Q1.E) |- (Q2.B)
            (Q1.C) |- (2,2)
            (Q2.C) |- (2,2)
            ;
        \end{circuitikz}
        \caption{Par Darlington}
        \label{fig:darlington}
    \end{minipage}\hfill
    \begin{minipage}[b]{0.48\textwidth}
        \centering
        \begin{circuitikz}[american voltages]
            \ctikzset{resistors/scale=0.5};
            \draw
            (-1.5,0) node[nigfete, xscale=-1](Q3){\scalebox{-1}[1]{$Q_3$}}
            (1.5,0) node[nigfete](Q4){$Q_4$}
            (Q3.G) -- (Q4.G)
            (1.5,2.5) node[vcc]{$V_{DD}$} to[R=$R_{ref}$, i=$I_{ref}$] (Q4.D) 
            (-1.5,2.5) node[anchor=east]{$V_o$} to[short,o-, i=$I_o$] (Q3.D)  
            (Q4.D) -| (0,-0.25) node[anchor=north](B){}
            (Q3.S) -- ++(0,-0.5) node[ground]{}
            (Q4.S) -- ++(0,-0.5) node[ground]{}
            ;
        \end{circuitikz}
        \caption{Espejo de Corriente n-MOSFET}
        \label{fig:nmos_mirror}
    \end{minipage}
\end{figure}

\section{Par Darlington}

El Par Darlington consiste en una configuración en Colector Común entre dos Transistores Bipolares de Juntura (BJT) y en cascada como se muestra en la figura \ref{fig:darlington}. Al conectar el emisor de $Q_1$ a la base de $Q_2$ la ganancia total de corriente de este circuito es el producto de las ganancias de corriente de cada transistor de manera que:
\begin{equation}
    I_{E2} = (h_{fe1} + 1 )(h_{fe2} + 1) I_{B1}
\end{equation}
donde $\beta_i$ es la ganancia de corriente de cada BJT.

Para este trabajo se utilizaron dos transistores BC547 con características \dots

\section{Espejo de Corriente MOSFET}

El Espejo de Corriente permite forzar la corriente a la salida del dispositivo a una corriente de referencia determinada. Para este trabajo se lo implemento con un par de transistores n-mos de enriquecimiento, configurados como en la figura \ref{fig:nmos_mirror}.

\section{Par Darlington con Carga Activa}
Uniendo los componentes mencionados previamente se pasó a diseñar y analizar el circuito de la figura \ref{fig:circuito_activo}. Como se observa, se trata de el Par Darlington actuando como circuito amplificador y el espejo de corriente con n-mos actuando como carga activa, fijando el punto de trabajo de los BJT del Par Darlington. De esta manera también debería presentar mayor estabilidad frente a cambios de tensión y temperatura.

En contraste, también se analizó el mismo circuito con una carga pasiva, como se ve en la figura \ref{fig:circuito_pasivo}, de forma de evaluar los beneficios de utilizar una carga activa.

\begin{figure} [ht]
    \centering
    \begin{minipage}[b]{0.44\textwidth}
        \centering
        \begin{circuitikz}
            \ctikzset{resistors/scale=0.5};
            \draw
            (7,6.5) node[vcc](Vcc){$V_{CC}$}
            (5.5,5) node[npn](Q1){$Q_1$}
            (7,4) node[npn](Q2){$Q_2$}
            (Q1.E) |- (Q2.B)
            (Q1.C) |- (Vcc) (Q2.C) |- (Vcc)
            (Q1.B) to[R=$R_B$,*-] ++(0,1.5) --(Vcc)
            (Q1.B) to[C=$C_i$] ++(-1,0) to[R=$R_s$] ++(-2,0) to[sV,l=$v_s$] ++(0,-2) node[ground]{}
            (Q2.E) to[short,-*] (7,2.5)
            (7,2.5) to[C=$C_L$] ++(-2,0) node[anchor=east]{$v_o$} to[R=$R_L$,o-] ++(0,-2) node[ground]{}
            (7,2.5) to[R=$R_E$] ++(0,-2) node[ground]{} 
            ;
        \end{circuitikz}
        \caption{Par Darlington con Carga Pasiva}
        \label{fig:circuito_pasivo}
    \end{minipage}\hfill
    \begin{minipage}[b]{0.52\textwidth}
        \centering
        \begin{circuitikz}
            \ctikzset{resistors/scale=0.5};
            \draw
            (10,6.5) node[vcc](Vcc){$V_{CC}$}
            (5.5,5) node[npn](Q1){$Q_1$}
            (7,4) node[npn](Q2){$Q_2$}
            (7,1) node[nigfete, xscale=-1](Q3){\scalebox{-1}[1]{$Q_3$}}
            (10,1) node[nigfete](Q4){$Q_4$}
            (Q3.S) node[ground]{} (Q4.S) node[ground]{}
            (Q1.E) |- (Q2.B)
            (Q1.C) |- (Vcc) (Q2.C) |- (Vcc)
            (Q4.D) to[R=$R_{ref}$] ++(0,4.75)
            (Q4.D) -| (8.5,0.75) (Q3.G) to[short,-*] (8.5,0.75) (Q4.G) to[short,-*] (8.5,0.75)
            (Q1.B) to[R=$R_B$,*-] ++(0,1.5) --(Vcc)
            (Q1.B) to[C=$C_i$] ++(-2,0) to[R=$R_s$] ++(0,-2) to[sV,l=$v_s$] ++(0,-2) node[ground]{}
            (Q2.E) to[short,-o] (7,2.5) node[anchor=west](vo){$v_o$} to[short] (Q3.D)
            (7,2.5) to[C,l=$C_L$] ++(-2,0) to[R=$R_L$] ++(0,-2) node[ground]{}
            ;
        \end{circuitikz}
        \caption{Par Darlington con Carga Activa}
        \label{fig:circuito_activo}
    \end{minipage}
\end{figure}

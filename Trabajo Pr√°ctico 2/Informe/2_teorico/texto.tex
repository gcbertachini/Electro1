\chapter{Análisis Teórico}
Se analizó el comportamento del circuito a pequeñas señales.
\section{Circuito de Polarización}
Como primer paso se analizó la polarización del circuito, ambos en su forma de carga pasiva (figura \ref{fig:pol_pasivo}) y activa (figura \ref{fig:pol_activo}).

\begin{figure} [ht]
    \centering
    \begin{minipage}{0.48\textwidth}
        \centering
        \begin{circuitikz}
            \ctikzset{resistors/scale=0.5};
            \draw
            (7,6.5) node[vcc](Vcc){$V_{CC}$}
            (5.5,5) node[npn](Q1){$Q_1$}
            (7,4) node[npn](Q2){$Q_2$}
            (Q1.E) |- (Q2.B)
            (Q1.C) |- (Vcc) (Q2.C) |- (Vcc)
            (Q1.B) to[R=$R_B$] ++(0,1.5) --(Vcc)
            (Q2.E) to[R=$R_E$] ++(0,-1) node[ground]{}
            ;
        \end{circuitikz}
        \caption{Circuito de Polarización con Carga Pasiva}
        \label{fig:pol_pasivo}
    \end{minipage}\hfill
    \begin{minipage}{0.48\textwidth}
        \centering
        \begin{circuitikz}
            \ctikzset{resistors/scale=0.5};
            \draw
            (7,6.5) node[vcc](Vcc){$V_{CC}$}
            (5.5,5) node[npn](Q1){$Q_1$}
            (7,4) node[npn](Q2){$Q_2$}
            (Q1.E) |- (Q2.B)
            (Q1.C) |- (Vcc) (Q2.C) |- (Vcc)
            (Q1.B) to[R=$R_B$] ++(0,1.5) --(Vcc)
            (Q2.E) to[I, l=$I_{ref}$] ++(0,-1) node[ground]{}
            ;
        \end{circuitikz}
        \caption{Circuito de Polarización con Carga Activa}
        \label{fig:pol_activo}
    \end{minipage}
\end{figure}

\subsection{Carga Pasiva}
En el circuito con Carga Pasiva, se resolvió analizando las mallas de entrada de los transistores.

\begin{align*}
    & V_{CC}-I_{B1}R_{B}-V_{BEon1}-V_{BEon2}-I_{E2}R_{E}=0 \\
    & I_{E2} = \left(1 + h_{FE1}\right)\left(1 + h_{FE2}\right) I_{B1} \\
    \Rightarrow & I_{E2} = \frac{V_{CC}-V_{BEon1}-V_{BEon2}}{R_E+\frac{R_B}{\left(1 + h_{FE1}\right)\left(1 + h_{FE2}\right)}}
\end{align*}

Considerando que la influencia de de $R_B$ será despreciable frente a la de $R_E$, y que ambos transistores tienen la misma $V_{BEon}$ se puede aproximar \eqref{eq:pol_I_E2}. Luego se pueden obtener las $I_{CQ}$ de cada transistor.

\begin{equation}
    I_{E2} \approx \frac{V_{CC}-2V_{BEon}}{R_E}
    \label{eq:pol_I_E2}
\end{equation}

\begin{align}
    I_{CQ2} &= \frac{h_{FE2}}{h_{FE2}+1}\cdot I_{E2} \label{eq:icq2} \\ 
    I_{CQ1} &= \frac{h_{FE1}}{h_{FE1}+1}\cdot \frac{1}{h_{FE2}+1}\cdot I_{E2} \label{eq:icq1}
\end{align}

Recorriendo las mallas de salida se obtuvieron las $V_{CEQ}$ de cada transistor.

\begin{align}
    V_{CEQ2} &= V_{CC} - I_{E2} R_E \\
    V_{CEQ1} &= V_{CC} - V_{BEon} - I_{E2} R_E
\end{align}

\subsection{Carga Activa}

A partir de la ecuación de Sah \eqref{eq:sah} para cada uno de los transistores se puede encontrar la razón entre la corriente de salida y de referencia.
\begin{equation}
    I_D = K' \frac{W}{L}\cdot (V_{GS}-V_{TH})^2
    \label{eq:sah}
\end{equation}

Considerando que por la configuración la tensión $V_{GS}$ será la misma para ambos transistores, y aproximando que, dado que se utilizarán transistores del mismo modelo, se puede aproximar que 

\begin{equation}
    I_o = \frac{(W/L)_3}{(W/L)_4} \cdot I_{ref}
\end{equation}

Por otro lado, puede diseñarse la $I_{ref}$ recorriendo la malla de $Q_4$

\begin{equation}
    I_{ref}=\frac{V_{DD}-V_{GS}}{R_{ref}}
    \label{eq:I_ref}
\end{equation}

Luego, los resultados de \eqref{eq:I_ref} pueden utilizarse en \eqref{eq:pol_I_E2}, \eqref{eq:icq2} y \eqref{eq:icq1}.

También a partir de la ecuación	 \eqref{eq:sah} se puede obtener las tensiones de polarización:

\begin{align}
    V_{DS} &= \sqrt{\frac{I_D L}{K' W}} + V_{TH}\\
    V_{CEQ2} &= V_{CC} - V_{DS} \\
    V_{CEQ1} &= V_{CC} - V_{DS} - V_{BEon}
\end{align}

\section{Parámetros de Pequeña Señal}

Se estimaron los parámetros de pequeña señal del para modelo de Giacoletto:

\begin{table}
    \centering
    \begin{tabular}{|c|c|c|}
        
    \end{tabular}
\end{table}


\section{Circuito Incremental}
Lorem ipsum
\subsection{Ganancia de Tensión}
\subsection{Ganancia de Corriente}
Lorem ipsum
\subsection{Impedancias de Entrada y Salida}
Lorem ipsum
\subsection{Respuesta en Frecuencia}
Lorem ipsum

\section{Selección de Componentes}
\subsection{Limitaciones de la Plataforma de Medición} \label{sec:EE_limits}

Dado que el \textit{Electronics Explorer} sobre el cual se hicieron las mediciones tiene un límite de corriente de $1.5 \si{\ampere}$ en las salidas programables, se diseñaron los circuitos de forma de no saturar al dispositivo. La salida $V_{cc}$ no tiene limitación de corriente, pero la tensión es muy baja.

Como se planea utilizar un espejo de corriente, donde se consideró que ambos transistores n-mos son idénticos, las corrientes de polarización que se le pedirán a la fuente estarán dadas por:

\begin{align*}
    I_{CQ} &= \frac{h_{fe}}{h_{fe}+1} \cdot I_{EQ} \\
    I_{BQ} &= \frac{1}{h_{fe}+1} \cdot I_{EQ}\\
    I_{CQ2} &= I_o = I_{ref} \\
    I &= I_{ref} + I_{CQ2} + I_{CQ1} + I_{BQ1}
\end{align*}

\begin{equation}
    \Rightarrow I = I_{ref} \left(1 + \frac{h_{fe2}}{h_{fe2}+1} + \frac{1}{h_{fe2}+1}\cdot\frac{h_{fe1}}{h_{fe1}+1}+\frac{1}{h_{fe2}+1}\cdot\frac{1}{h_{fe1}+1}\right)
\end{equation}

Aproximando a valores altos de ganancias de corriente,

\begin{equation}
    I \approx I_{ref}\left(2+1/h_{fe2}\right)
\end{equation}

A partir de la expresión anterior se consideró entonces una cota máxima de la corriente de referencia:
\begin{equation}
    I_{ref} < 0.75 \si{\ampere}
    \label{eq:Iref}
\end{equation}
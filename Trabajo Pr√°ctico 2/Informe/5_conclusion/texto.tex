\chapter{Conclusión}
En primer lugar, se puede observar claramente la zona de trabajos en frecuencias medias en las simulaciones obtenidas, en especial para la 
ganancia de tensión. Para dichas frecuencias los capacitores externos ya se los puede considerar en cortocircuito, pero todavía los capacitores 
internos parásitos no tienen mucha preponderancia. La zona de frecuencias medias para este circuito se puede observar entre aproximadamente 
10 $KHz$ y 10 $MHz$.

Por otro lado, al ser el Darlington un amplificador de corriente, y considerando las limitaciones propias de la placa Digilent Explorer,
 resultó prácticamente imposible medir con precisión la corriente de entrada del  sistema, siendo sólo posible estimar con mucho error la misma,
  obteniendo diferencias importantes en la experiencia empírica respecto de
 lo teórico y simulado.


Teóricamente se estimó que la aplicación de una carga activa en lugar de una pasiva significa una mejora en la ganancia unitaria deseada del circuito amplificador de corriente. Esto significa una mejora en la estabilidad del circuito, que se puede observar en una menor varianzia en primer lugar de las frecuencias de los polos de alta frecuencia en las simulaciones estadísticas.

A su vez, aumenta considerablemente la impedancia de entrada del par darlington (aún cuando la impedancia del circuito amplificador aún se mantiene limitado por la resistencia de base que sea escogida).

Preliminarmente se verificó en la práctica una mejora en la ganancia unitaria de tensión del circuito al cambiar de una carga pasiva a una carga activa.
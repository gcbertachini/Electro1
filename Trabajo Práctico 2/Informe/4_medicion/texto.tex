\chapter{Mediciones y Resultados}
Se conectan entonces ambos circuitos en la placa Electronics Explorer. Debido a las limitaciones del dispositivo, se utilizó una tension de alimentacion de $V_{CC} = 9V$ tanto para el par Darlington como para la carga activa.
A su vez se utiliza el generador arbitrario de funciones para inyectar una señal de $500 mV$ a una frecuencia de $10kHz$ con el objetivo de medir la respuesta del circuito a pequenias seniales. En principio se plantea una senal de $50 mV$, pero debido a una considerable cantidad de ruido en la medicion se imposibilita la medicion precisa del sistema.
Requiriendo entonces una mayor entrada, esta se incrementa cuidando no perder la linealidad de la respuesta hasta el valor propuesto previamente.
\section{Circuito Construido}
El circuito realizado se encuentra en la figura REFERENCIA. En donde se alterna en el emisor de $Q2$ entre dos resistencias en serie representando $R_E$ y en la fuente de corriente espejo conformado por los MOS-FET del lado derecho.

INSERTAR FOTO


\section{Mediciones}
A continuacion se presentan las mediciones de los parametros presentados previamente de manera teorica. Tanto para el Darlington con carga activa y carga pasiva.

\subsection{Ganancia de Tensión}

En la tabla REFERENCIA se muestran los resultados para $\Delta_V$ y $\Delta_{Vs}$ en ambas configuraciones. Y en contraste con los simulado y lo esperado teoricamente.

INSERTAR TABLA 

Tambien se observan en las figuras REFERENCIAS, el contraste entre formas de onda de salida y entrada para ambas cargas.

INSERTAR FIGURAS.

\subsection{Ganancia de Corriente}
%% No la medimos por que la de base es muy chica. Se podria calcular la de salida con los daos de AVS
\subsection{Impedancias de Entrada y Salida}

El procedimiento que se emplea para medir la impedancia de entrada consta en insertar una resistencia de referencia de $1 k\Omega$ con tolerancia $1\%$entre el generador de ondas y el capacitor de desacople en la base de $Q1$. Se elige este valor ya que es comparable al esperado teoricamente (ESTO NOSE BIEN GUITARREARLO).
Midiendo la caida de tension en esta y dividiendo por su valor nominal, se haya la corriente que entra a la base del transistor para pequenias seniales. Luego el cociente entre la tension de entrada a la base, y la corriente previamente calculada se estima la impedancia de entrada.
Se realiza el mismo procedimiento para las dos cargas, por mas que se espera un resultado muy similar. En la tabla REFERENCIA, se contrasta lo medido,simulado y calculado teoricamente.

TABLA  

Para la impedancia de salida se realiza una tecnica prevista en el material didactico de la catedra. Consta en medir la tension a la salida del amplificador a circuito abierto (carga infinita),denominada $V_{open}$. Luego se introduce una carga $R_L$ y se mide la tension $V_L$ a traves de la misma.
Contando con estos valores se emplea la ecuacion \ref{eq:zout teor}, encontrando asi un valor para la impedancia de salida.

\begin{equation}
    R_O = R_L(\frac{V_{open}}{V_L}-1)
    \label{eq:zout teor}
\end{equation}

\subsection{Respuesta en Frecuencia}

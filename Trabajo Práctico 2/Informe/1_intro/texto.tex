\chapter{Introducción}

El objetivo de este trabajo fue comprobar en la práctica algunos de los aspectos más destacados de un circuito estudiado durante la cátedra de Electrónica I, dentro del marco de una simulación con recursos escasos.

En este caso, se estudió el comportamiento de un Par Darlington con una carga activa, en la forma de un espejo de corriente implementado con un par de transistores MOSFET.

\section{Par Darlington}

\begin{figure}[ht]
    \begin{center}
        \begin{circuitikz}[american voltages]
            \draw
            (0,1) node[npn](Q1){$Q_1$}
            (1,0) node[npn](Q2){$Q_2$}
            (Q1.E) |- (Q2.B)
            (Q1.C) |- (2,2)
            (Q2.C) |- (2,2)
            ;
        \end{circuitikz}
        \caption{Par Darlington}
    \end{center}
\end{figure}

El Par Darlington consiste en una conexión en cascada entre dos transistores bipolares de juntura (BJT de sus siglas en inglés). Al conectar el emisor del primer transistor a la base del segundo, la corriente entrante a la base del primer transistor es multiplicada por las ganancias de ambos transistores (aproximadamente)

\section{Espejo de Corriente MOSFET}

\begin{figure}[ht]
    \begin{center}
        \begin{circuitikz}[american voltages]
            \ctikzset{resistors/scale=0.5};
            \draw
            (-1.5,0) node[nigfete, xscale=-1](Q3){\scalebox{-1}[1]{$Q_3$}}
            (1.5,0) node[nigfete](Q4){$Q_4$}
            (Q3.G) -- (Q4.G)
            (1.5,3) node[vcc]{$V_{DD}$} to[R=$R_{ref}$, i=$I_{ref}$] (Q4.D) 
            (-1.5,3) node[anchor=east]{$V_o$} to[short,o-, i=$I_o$] (Q3.D)  
            (Q4.D) -| (0,-0.25)
            (Q3.S) -- ++(0,-0.5) node[ground]{}
            (Q4.S) -- ++(0,-0.5) node[ground]{}
            ;
        \end{circuitikz}
        \caption{Par Darlington}
    \end{center}
\end{figure}
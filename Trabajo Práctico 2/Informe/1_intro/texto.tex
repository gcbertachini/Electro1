\chapter{Introducción}

El objetivo de este trabajo fue comprobar en la práctica algunos de los aspectos más destacados de un circuito estudiado durante la cátedra de Electrónica I, dentro del marco de una simulación con recursos escasos.

En este caso, se estudió el comportamiento de un Par Darlington con una carga activa, en la forma de un espejo de corriente implementado con un par de transistores MOSFET.

\section{Par Darlington}

\begin{figure}[ht]
    \begin{center}
        \begin{circuitikz}[american voltages]
            \draw
            (0,1) node[npn](Q1){$Q_1$}
            (1,0) node[npn](Q2){$Q_2$}
            (Q1.E) |- (Q2.B)
            (Q1.C) |- (2,2)
            (Q2.C) |- (2,2)
            ;
        \end{circuitikz}
        \caption{Par Darlington}
        \label{fig:darlington}
    \end{center}
\end{figure}

El Par Darlington consiste en una configuración en Colector Común entre dos Transistores Bipolares de Juntura (BJT) y en cascada como se muestra en la figura \ref{fig:darlington}. Al conectar el emisor de $Q_1$ a la base de $Q_2$ la ganancia total de corriente de este circuito es el producto de las ganancias de corriente de cada transistor de manera que:
\begin{equation}
    I_{E2} = (\beta_1 + 1 )(\beta_2 + 1) I_{B1}
\end{equation}
donde $\beta_i$ es la ganancia de corriente de cada BJT.

Para este trabajo se utilizaron dos transistores BC547 con características \dots

\section{Espejo de Corriente MOSFET}

El Espejo de Corriente permite forzar la corriente a la salida del dispositivo a una corriente de referencia determinada. Para este trabajo se lo implemento con un par de transistores n-mos de enriquecimiento, configurados como en la figura \ref{fig:nmos_mirror}.

\begin{figure}[ht]
    \begin{center}
        \begin{circuitikz}[american voltages]
            \ctikzset{resistors/scale=0.5};
            \draw
            (-1.5,0) node[nigfete, xscale=-1](Q3){\scalebox{-1}[1]{$Q_3$}}
            (1.5,0) node[nigfete](Q4){$Q_4$}
            (Q3.G) -- (Q4.G)
            (1.5,2.5) node[vcc]{$V_{DD}$} to[R=$R_{ref}$, i=$I_{ref}$] (Q4.D) 
            (-1.5,2.5) node[anchor=east]{$V_o$} to[short,o-, i=$I_o$] (Q3.D)  
            (Q4.D) -| (0,-0.25) node[anchor=north](B){}
            (Q3.S) -- ++(0,-0.5) node[ground]{}
            (Q4.S) -- ++(0,-0.5) node[ground]{}
            ;
        \end{circuitikz}
        \caption{Espejo de Corriente n-MOSFET}
        \label{fig:nmos_mirror}
    \end{center}
\end{figure}

A partir de la ecuación de Sah \eqref{eq:sah} para cada uno de los transistores se puede encontrar la razón entre la corriente de salida y de referencia.
\begin{equation}
    I_D = K' \frac{W}{L}\cdot (V_{GS}-V_{TH})^2
    \label{eq:sah}
\end{equation}

Considerando que por la configuración la tensión $V_{GS}$ será la misma para ambos transistores, y aproximando que, dado que se utilizarán transistores del mismo modelo, se puede aproximar que 

\begin{equation}
    I_o = \frac{(W/L)_3}{(W/L)_4} \cdot I_{ref}
\end{equation}

Por otro lado, puede diseñarse la $I_{ref}$ recorriendo la malla de $Q_4$

\begin{equation}
    I_{ref}=\frac{V_{DD}-V_{GS}}{R_{ref}}
\end{equation}

\section{Par Darlington con Carga Activa}
Finalmente, el circuito a diseñar y analizar resulta en el mostrado en la figura \ref{fig:circuito}

\begin{figure}[ht]
    \begin{center}
        \begin{circuitikz}
            \ctikzset{resistors/scale=0.5};
            \draw
            (7,2.5) node[anchor=west](vo){$v_o$}
            (11,6.5) node[vcc](Vcc){$V_{CC}$}
            (5.5,5) node[npn](Q1){$Q_1$}
            (7,4) node[npn](Q2){$Q_2$}
            (7,1) node[nigfete, xscale=-1](Q3){\scalebox{-1}[1]{$Q_3$}}
            (10,1) node[nigfete](Q4){$Q_4$}
            (Q3.S) node[ground]{} (Q4.S) node[ground]{}
            (Q1.E) |- (Q2.B)
            (Q1.C) |- (Vcc) (Q2.C) |- (Vcc)
            (Q4.D) to[R=$R_{ref}$] ++(0,4.75)
            (Q4.D) -| (8.5,0.75) (Q3.G) to[short,-*] (8.5,0.75) (Q4.G) to[short,-*] (8.5,0.75)
            (Q1.B) to[R=$R_B$,*-] ++(0,1.5) --(Vcc)
            (Q1.B) to[C=$C_i$] ++(-1,0) to[R=$R_s$] ++(-2,0) to[sV,l=$v_i$] ++(0,-2) node[ground]{}
            (Q2.E) to[short,-o] (7,2.5) (Q3.D) to[short] (7,2.5)
            (7,2.5) to[short] ++(-2,0) to[R=$R_L$] ++(0,-2) node[ground]{}
            ;
        \end{circuitikz}
        \caption{Par Darlington con Carga Activa}
        \label{fig:circuito}
    \end{center}
\end{figure}


\section{Limitaciones de la Plataforma de Medición}

Dado que el \textit{Electronics Explorer} sobre el cual se hicieron las mediciones tiene un límite de corriente de $1.5 \si{\ampere}$ en las salidas programables, se diseñaron los circuitos de forma de no saturar al dispositivo. La salida $V_{cc}$ no tiene limitación de corriente, pero la tensión es muy baja.